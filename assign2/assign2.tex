\documentclass[11pt,epsfig]{article}
\usepackage{hyperref}
\usepackage{geometry}
\usepackage{url}
\newcommand{\cM}{{\cal M}} 
\newcommand{\cC}{{\cal C}}
\newcommand{\cK}{{\cal K}} 
\newcommand{\bZ}{{\bf Z}}

\setlength{\topmargin}{0in}
\setlength{\leftmargin}{0in}
\setlength{\textwidth}{6.25in}
\setlength{\textheight}{9in}
\setlength{\parindent}{0.2in}
\setlength{\parskip}{.2in}
\voffset = -.75in
\hoffset = -.79in
%\pagestyle{empty}
\def\floor#1{\left\lfloor #1\right\rfloor} %  floor and ceiling (from math mode)
\def\ceil#1{\left\lceil #1\right\rceil} 

\def\ang#1{\langle #1\rangle}	%  angle brackets <...> 


 
\begin{document}
\newcommand{\lsp}[1]{\large\renewcommand{\baselinestretch}{#1}\normalsize}
\lsp{1}
{\bf 
\begin{center}
COMP5631: Cryptography and Security \\
2025 Spring -- Written Assignment Number 2 \\
Handed out: on Feb. 21, 2023 \\ 
Due: on March 9 by 11:30pm. \\ 

\end{center}
}

\thispagestyle{empty} \pagestyle{empty}

%\noindent{\it Assignments handed in during class will lose marks.
%              No assignments will be accepted after class. No email 
%              submission will be accepted.}

\noindent{\it Please upload your solution paper into Canvas by 23:30 on March 9. 
No email 
              submission will be accepted.}

\begin{description}


\item[Q1.] 
Please read the transposition  cipher documented in the Appendix of Lecture 2. 
Then use the example transposition cipher on Slide No. 25 to encrypt the message 
\texttt{killthem}. Write down the corresponding ciphertext.  \hfill \fbox{10 marks} 

\textbf{Solution:}



\item[Q2.] 
There are ten pieces of ciphertext in the following URL:  \\
 \href{https://home.cse.ust.hk/faculty/cding/COMP5631/c685-5.html}{[Click here]} \\
 Each of them is obtained by encrypting an English article with a simple substitution 
 cipher. Let $\ell$ denote the last digit in your student ID number, please then choose the 
 $((\ell +1) \bmod{10})$-th ciphertext and recover the original plaintext. 
 
 
 You may write your own computer programs or use the following online software to compute the frequencies of single 
 letters, digraphs and trigraphs in the ciphertext for you: 
 \begin{verbatim} 
 https://www.cryptoclub.org/#vCiphers
 \end{verbatim} 
 Please write down certain details of your decryption process. You need to write down your decrypted text (i.e., 
 the readable text), but do not need to write down the one-to-one function used for encryption.    
\hfill \fbox{30 marks}


\item[Q3.]
Consider the example cipher on Slide No. 22 of Lecture 3, where $p$ is a prime.  Let 
$\ell =2$ (i.e., the cipher has two rounds of iteration of the 
round function $f_h(x)$).  Write down the encryption function $E_k(m)$ and decryption function $D_k(c)$ of this cipher.     \hfill \fbox{20 marks} 


\item[Q4.] 
Given a one-key block cipher $(\cM, \cC, \cK, E_k, D_{k})$, where $\cM=\cC$ and $E_k$ 
maps an $n$-bit block into an $n$-bit block, we can 
construct a new one-key block cipher by picking up two keys $k_1$ and $k_2$ for the 
original cipher to form a key $k=(k_1,k_2)$ for the new block cipher. The 
encryption and description of the new cipher go as follows: 
\begin{description}
\item[Encryption:] $c=E_{k_2}(E_{k_1}(m))$. 
\item[Decryption:] $m=D_{k_1}(D_{k_2}(c))$.  
\end{description}
Thus the new block cipher has the same block length as the original cipher, 
but its key length doubles that of the original cipher. This is the double 
encryption introduced in Lecture 5. 

Design a specific one-key cipher and show that double-encryption with 
this cipher does not increase the security level of the original cipher at all. 
[Hint: Consider some of the ciphers on some lecture sides.] \hfill \fbox{20 marks}



\item[Q5.] 
 Show that the Diffie-Hellman Key Exchange (Agreement) Protocol is insecure with respect to active attacks.    \hfill \fbox{20 marks}






\end{description} 


\end{document}



